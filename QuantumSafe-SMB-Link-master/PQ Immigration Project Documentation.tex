\documentclass[12pt,oneside]{report}

% =========================================
%          PACKAGES & BASIC SETUP
% =========================================

\usepackage[a4paper,margin=3cm]{geometry}
\usepackage{setspace}
\onehalfspacing

\usepackage[T1]{fontenc}
\usepackage[utf8]{inputenc}
\usepackage{lmodern} % or Times if your uni requires it
%\usepackage{times}

\usepackage{amsmath,amssymb,amsthm}
\usepackage[utf8]{inputenc}
\usepackage{mathtools}
\usepackage{siunitx}
\usepackage{booktabs}
\usepackage{multirow}
\usepackage{graphicx}
\usepackage{float}
\usepackage{caption}
\usepackage{subcaption}
\usepackage{tikz}
\usetikzlibrary{arrows.meta, positioning, shapes.misc}
\usepackage{listings}
\usepackage{color}
\usepackage{url}
\usepackage{hyperref}
\usepackage[noabbrev,nameinlink]{cleveref}
\usepackage{tocbibind} % include refs in TOC

% =========================================
%                 HYPERREF
% =========================================

\hypersetup{
    colorlinks=true,
    linkcolor=black,
    citecolor=black,
    urlcolor=blue,
    pdftitle={QuantumSafe-SMB-Link: Post-Quantum Hybrid SMB Tunnel},
    pdfauthor={Your Name},
    pdfcreator={LaTeX}
}

% =========================================
%            LISTINGS FOR RUST
% =========================================

\definecolor{lightgray}{gray}{0.95}
\definecolor{darkgray}{gray}{0.25}
\definecolor{purple}{rgb}{0.5,0,0.5}
\definecolor{myblue}{rgb}{0,0,0.5}

\lstdefinelanguage{Rust}{
  keywords={break, continue, else, for, if, in, loop, match, while,
    as, const, fn, let, move, mut, ref, return, static, struct, trait,
    type, unsafe, use, where, crate, impl, dyn, pub, enum, async, await},
  keywordstyle=\color{purple}\bfseries,
  ndkeywords={u8, u16, u32, u64, usize, i32, i64, bool, String, Vec, Option, Result, Box, Arc, TcpStream},
  ndkeywordstyle=\color{myblue}\bfseries,
  identifierstyle=\color{black},
  sensitive=true,
  comment=[l]{//},
  morecomment=[s]{/*}{*/},
  commentstyle=\color{darkgray}\itshape,
  stringstyle=\color{blue},
  morestring=[b]",
}

\lstdefinestyle{ruststyle}{
    language=Rust,
    basicstyle=\ttfamily\small,
    backgroundcolor=\color{lightgray},
    showstringspaces=false,
    tabsize=4,
    breaklines=true,
    frame=single,
    framerule=0.3pt
}

\lstset{style=ruststyle}

% =========================================
%          THEOREM-LIKE ENVIRONMENTS
% =========================================

\newtheorem{definition}{Definition}[chapter]
\newtheorem{theorem}{Theorem}[chapter]
\newtheorem{lemma}{Lemma}[chapter]

% =========================================
%                  MACROS
% =========================================

\newcommand{\Kyber}{\textsf{ML-KEM-768}}
\newcommand{\Dilithium}{\textsf{ML-DSA-2}}
\newcommand{\XDH}{\textsf{X25519}}
\newcommand{\ChaCha}{\textsf{ChaCha20-Poly1305}}
\newcommand{\qsl}{\textsf{QuantumSafe-SMB-Link}}

% =========================================
%                DOCUMENT
% =========================================

\begin{document}

% =========================================
%               TITLE PAGE
% =========================================

\begin{titlepage}
    \centering
    \vspace*{2cm}
    {\Large Applied Cryptography -- CMPS 297AD/396AI \par}
    \vspace{1cm}
    {\huge\bfseries QuantumSafe-SMB-Link\par}
    \vspace{0.4cm}
    {\Large A Post-Quantum Hybrid Tunnel for SMB\par}
    \vspace{2.5cm}

    {\Large Your Name\par}
    \vspace{0.5cm}
    {\large Department of Computer Science\par}
    {\large American University of Beirut\par}

    \vfill

    {\large \today\par}
\end{titlepage}

\pagenumbering{roman}

% =========================================
%                 ABSTRACT
% =========================================

\chapter*{Abstract}
\addcontentsline{toc}{chapter}{Abstract}

The advent of large-scale quantum computers threatens the public-key cryptography currently used to secure network protocols and tunnels. In particular, elliptic-curve Diffie--Hellman (ECDH) and RSA, which protect many VPNs and SMB tunnels, can be broken by Shor's algorithm. To address this risk, NIST has standardized lattice-based post-quantum primitives, including \Kyber{} for key encapsulation and \Dilithium{} for digital signatures.

This project presents \emph{\qsl{}}, a Rust implementation of a hybrid, post-quantum-aware tunnel for the Server Message Block (SMB) protocol. On top of TCP, \qsl{} runs a custom handshake that negotiates capabilities, performs a hybrid key establishment using \Kyber{} and optional \XDH{}, and derives symmetric keys via an HKDF-based key schedule. SMB traffic is then carried inside a simple encrypted framing layer protected by the \ChaCha{} authenticated-encryption scheme. Detached \Dilithium{} signatures are implemented and can be used to authenticate handshake transcripts.

The report documents the protocol design, the Rust implementation architecture, and the migration rationale from classical to hybrid post-quantum security. It also analyzes performance and security trade-offs, discusses downgrade resistance and hybrid security guarantees, and outlines how this design can guide real-world migration of SMB-based infrastructures to post-quantum cryptography.

% =========================================
%               ACKNOWLEDGMENTS
% =========================================

\chapter*{Acknowledgments}
\addcontentsline{toc}{chapter}{Acknowledgments}

I would like to thank the course instructor, Dr.\ Nadim Kobeissi, for designing a project that bridges theoretical post-quantum cryptography with practical software engineering, and for the guidance provided throughout the semester.

\tableofcontents
\listoffigures
\listoftables

\clearpage
\pagenumbering{arabic}

% =========================================
%           CHAPTER 1: INTRODUCTION
% =========================================

\chapter{Introduction}
\label{chap:intro}

\section{Motivation}

Server Message Block (SMB) is widely used in organizational networks for file sharing, printer sharing, and networked storage. In many deployments, SMB traffic is either implicitly trusted within a local network or protected using classical VPNs that rely on elliptic-curve Diffie--Hellman (ECDH) or RSA. These classical public-key schemes are vulnerable to quantum attacks: a sufficiently powerful quantum computer running Shor's algorithm could recover private keys and decrypt recorded traffic.

The threat is particularly relevant for \emph{record-now, decrypt-later} adversaries, who capture encrypted data today with the expectation of decrypting it once quantum computers mature. SMB traffic often contains sensitive files and credentials with long-term confidentiality requirements, making it an important candidate for post-quantum migration.

At the same time, replacing widely deployed protocols is challenging. Full replacement of TLS or SMB stacks is not always feasible. Instead, many organizations rely on protocol-agnostic tunnels: programs that accept cleartext connections (e.g., SMB) and forward them through an encrypted channel. Migrating such tunnels to post-quantum security requires integrating new primitives, designing hybrid modes, and maintaining backward compatibility.

\section{Project Overview}

This project implements \emph{\qsl{}}, a proof-of-concept SMB tunnel that uses NIST-standardized post-quantum algorithms in a hybrid design. The tunnel is built in Rust and consists of:

\begin{itemize}
    \item A \textbf{hybrid handshake} that negotiates supported KEM and signature algorithms, performs \Kyber{} encapsulation, and optionally performs an \XDH{} ECDH exchange.
    \item A \textbf{key schedule} that combines classical and post-quantum shared secrets via HKDF over SHA-256 to derive bidirectional traffic keys.
    \item An \textbf{authenticated encryption layer} based on \ChaCha{}, using per-frame nonces and associated data (AAD).
    \item A \textbf{length-prefixed framing layer} that carries encrypted SMB payloads over an established TCP stream.
    \item A set of \textbf{Rust modules} that clearly separate networking, cryptographic operations, and protocol logic.
\end{itemize}

The codebase is organized to be educational: it exposes the protocol machinery explicitly, rather than hiding it behind generic TLS libraries. At the same time, the design borrows concepts from TLS~1.3 such as transcripts and HKDF-based key schedules.

\section{Contributions}

The contributions of this project are:

\begin{enumerate}
    \item \textbf{Protocol Design:} A concrete, documented wire protocol for a post-quantum / classical hybrid handshake aimed at SMB tunneling.
    \item \textbf{Implementation:} A full Rust implementation using production-quality crates for \Kyber{}, \Dilithium{}, \XDH{}, \ChaCha{}, and HKDF.
    \item \textbf{Documentation \& Analysis:} A structured explanation of the migration rationale, implementation details, performance considerations, and a security analysis of the design.
\end{enumerate}

\section{Document Structure}

The rest of this report is organized as follows:

\begin{itemize}
    \item \Cref{chap:background} covers the cryptographic background and the primitives used.
    \item \Cref{chap:requirements} defines system and security requirements.
    \item \Cref{chap:design} describes the protocol, handshake messages, and key schedule in detail.
    \item \Cref{chap:implementation} documents the Rust modules and key functions.
    \item \Cref{chap:performance} discusses performance and provides a structure for presenting benchmark results.
    \item \Cref{chap:security} analyzes the security of the hybrid design and discusses downgrade resistance and implementation concerns.
    \item \Cref{chap:conclusion} concludes and outlines future work.
    \item Appendices provide wire-format details and CLI usage examples.
\end{itemize}

% =========================================
%           CHAPTER 2: BACKGROUND
% =========================================

\chapter{Background}
\label{chap:background}

\section{Threat Model and Quantum Attacks}

We consider a standard Dolev--Yao network attacker with full control over the network: the adversary can intercept, delay, drop, modify, and inject packets. Additionally, we consider a \emph{record-now, decrypt-later} adversary who stores ciphertexts and later obtains access to a quantum computer.

Classical public-key schemes such as RSA, Diffie--Hellman, and elliptic-curve Diffie--Hellman rely on problems (factoring, discrete logarithms) that are efficiently solvable by quantum computers running Shor's algorithm. This breaks both confidentiality and authentication.

For SMB tunnels that protect long-lived or sensitive data, we therefore require key establishment mechanisms that remain secure in the presence of quantum attackers.

\section{Post-Quantum Primitives}

\subsection{ML-KEM-768 (\Kyber{})}

\Kyber{} is a lattice-based KEM selected by NIST for standardization. It provides three parameter sets; this project uses the \texttt{kyber768} variant via the \texttt{pqcrypto-kyber} crate. The main algorithms are:

\begin{itemize}
    \item \textsf{KeyGen}: $(pk, sk) \gets \textsf{KeyGen}()$.
    \item \textsf{Encaps}: Given $pk$, output $(ct, ss)$, where $ss$ is the shared secret.
    \item \textsf{Decaps}: Given $ct$ and $sk$, recover $ss$.
\end{itemize}

The shared secret $ss$ is used as part of the input keying material to HKDF.

\subsection{ML-DSA-2 (\Dilithium{})}

\Dilithium{} is a lattice-based signature scheme. This project uses the \texttt{dilithium2} variant via the \texttt{pqcrypto-dilithium} crate. The API (as wrapped by \texttt{sig.rs}) is:

\begin{itemize}
    \item \texttt{generate\_keys()} → (pk, sk)
    \item \texttt{sign\_detached(msg, sk)} → sig
    \item \texttt{verify\_detached(msg, sigma, pk)} → success/failure
 success/failure
\end{itemize}






The current codebase includes these helpers and tests; actual integration into the handshake for authentication is summarized in \Cref{chap:security} as future work.

\section{Classical Primitives}

\subsection{X25519 (\XDH{})}

For hybrid operation, the project uses X25519 via the \texttt{x25519-dalek} crate. Each side generates an ephemeral secret and public key:

\[
    \textsf{pk} = \textsf{X25519}(sk, G),
\]
and the shared secret is:

\[
    ss_{\text{classical}} = \textsf{X25519}(sk_{\text{local}}, pk_{\text{peer}}).
\]

\subsection{Authenticated Encryption (\ChaCha{})}

Tunnel data is protected with \ChaCha{} using the \texttt{chacha20poly1305} crate. It provides authenticated encryption with a 256-bit key and 96-bit nonce. In \texttt{aead.rs}, the functions are:

\begin{itemize}
    \item \texttt{aead\_seal(key, nonce, aad, plaintext)}.
    \item \texttt{aead\_open(key, nonce, aad, ciphertext)}.
\end{itemize}

\subsection{HKDF over SHA-256}

Key derivation is based on HKDF with SHA-256, implemented via the \texttt{hkdf} crate. The transcript hash is used as the salt; the concatenation of classical and PQ shared secrets is used as input keying material.

% =========================================
%        CHAPTER 3: SYSTEM REQUIREMENTS
% =========================================

\chapter{System and Protocol Requirements}
\label{chap:requirements}

\section{Functional Requirements}

The \qsl{} system is designed to satisfy the following functional requirements:

\begin{itemize}
    \item Act as an SMB tunnel: accept incoming SMB connections on a local address and forward them to a configured remote SMB server through an encrypted channel.
    \item Perform a key-establishment handshake before any SMB data is exchanged.
    \item Support both hybrid (\XDH{} + \Kyber{}) and pure post-quantum (\Kyber{}-only) modes.
    \item Provide a simple command-line interface to configure server/client roles, addresses, and cryptographic options.
\end{itemize}

\section{Security Requirements}

The security requirements are:

\begin{itemize}
    \item \textbf{Confidentiality:} SMB payloads must remain confidential against network attackers.
    \item \textbf{Integrity:} Modifications to tunnel traffic must be detected and rejected.
    \item \textbf{Forward secrecy:} Compromise of long-term keys must not reveal past session keys.
    \item \textbf{Post-quantum security:} Session keys should remain secure against quantum attacks due to the \Kyber{} component.
    \item \textbf{Downgrade resistance:} When both parties support PQ/hybrid suites, the protocol should not silently fall back to weaker classical-only modes.
\end{itemize}

\section{Non-Functional Requirements}

\begin{itemize}
    \item \textbf{Performance:} The handshake overhead must be acceptable, and steady-state tunnel throughput should approach that of raw TCP.
    \item \textbf{Portability:} The Rust code should compile and run on major operating systems supported by the Rust toolchain.
    \item \textbf{Robustness:} Invalid or malformed handshake messages must be handled gracefully, with precise error messages for debugging.
\end{itemize}

% =========================================
%          CHAPTER 4: PROTOCOL DESIGN
% =========================================

\chapter{Protocol Design}
\label{chap:design}

\section{High-Level Flow}

The \qsl{} protocol operates over a single TCP connection, with the following phases:

\begin{enumerate}
    \item \textbf{Handshake phase:} Client and server exchange \texttt{ClientHello} and \texttt{ServerHello} messages, negotiate a cryptographic suite, perform hybrid key establishment, and derive symmetric keys.
    \item \textbf{Data phase:} SMB payloads are encrypted, framed, and relayed between tunnel endpoints.
\end{enumerate}

\begin{figure}[H]
    \centering
    \begin{tikzpicture}[node distance=1.4cm, >=latex]
        \node (client) {Client};
        \node[right=6cm of client] (server) {Server};

        \draw[->] (client) -- node[above]{ClientHello} (server);
        \draw[->] (server) -- node[above]{ServerHello} (client);
        \draw[->] (client) -- node[above]{ClientAuth (optional)} (server);

        \node[below=1.8cm of client, text width=6cm, align=left] (cstate)
            {Client: derive classical/PQ shared secrets, compute transcript hash, derive keys};
        \node[below=1.8cm of server, text width=6cm, align=left] (sstate)
            {Server: derive classical/PQ shared secrets, compute transcript hash, derive keys};

        \draw[->] (client) -- (cstate);
        \draw[->] (server) -- (sstate);
    \end{tikzpicture}
    \caption{High-level handshake flow between client and server.}
    \label{fig:handshake-high}
\end{figure}

\section{Message Types and Magic Values}

\texttt{kem.rs} defines constant magic values and protocol version:

\begin{itemize}
    \item \texttt{CLIENT\_HELLO\_MAGIC = "QSLH"}.
    \item \texttt{SERVER\_HELLO\_MAGIC = "QSLS"}.
    \item \texttt{CLIENT\_AUTH\_MAGIC = "QSLA"}.
    \item \texttt{PROTOCOL\_VERSION = 1}.
\end{itemize}

Each handshake message starts with a magic constant and protocol version, followed by a length-prefixed payload. \Cref{app:wire} provides the binary layout.

\section{Capabilities and Algorithm Negotiation}

Algorithm capabilities are represented by the \texttt{Capabilities} struct:

\begin{itemize}
    \item A list of supported KEM algorithms (\Kyber{}, future variants).
    \item A list of supported signature algorithms (\Dilithium{}, etc.).
    \item A boolean \texttt{hybrid} flag.
    \item A boolean \texttt{pq\_required} flag.
\end{itemize}

These are encoded as TLV (type--length--value) pairs, with tags:

\begin{itemize}
    \item \texttt{TLV\_KEM\_LIST = 0x0001}.
    \item \texttt{TLV\_SIG\_LIST = 0x0002}.
    \item \texttt{TLV\_HYBRID = 0x0003}.
    \item \texttt{TLV\_PQ\_REQUIRED = 0x0004}.
\end{itemize}

The decoding logic checks alignment and length, rejecting malformed messages (e.g., KEM list not aligned to 2-byte IDs).

The server selects a \texttt{ChosenSuite} that belongs to the intersection of client and server capabilities. The \texttt{pq\_required} flag prevents the server from selecting a non-PQ suite when the client insists on post-quantum security.

\section{ClientHello and ServerHello}

\subsection{ClientHello Structure}

The \texttt{ClientHello} struct (simplified) is:

\begin{lstlisting}
pub struct ClientHello {
    pub random: [u8; 32],
    pub x_pub: [u8; 32],
    pub kyber_pub: Vec<u8>,
    pub caps: Capabilities,
    pub caps_encoded: Vec<u8>,
    // internal: optional EphemeralSecret and Kyber secret
}
\end{lstlisting}

\begin{itemize}
    \item \texttt{random} is filled with 32 bytes from \texttt{OsRng}.
    \item \texttt{x\_pub} is the client's X25519 public key (zeros if hybrid is disabled).
    \item \texttt{kyber\_pub} is the serialized \Kyber{} public key.
    \item \texttt{caps} and \texttt{caps\_encoded} represent capabilities and their binary encoding.
\end{itemize}

\subsection{ServerHello Structure}

The \texttt{ServerHello} struct is:

\begin{lstlisting}
pub struct ServerHello {
    pub random: [u8; 32],
    pub x_pub: [u8; 32],
    pub kyber_ct: Vec<u8>,
    pub chosen_suite: ChosenSuite,
    pub chosen_suite_encoded: Vec<u8>,
    pub signature: Vec<u8>,
}
\end{lstlisting}

\begin{itemize}
    \item \texttt{random} is server-generated.
    \item \texttt{x\_pub} is server X25519 public key (zeros when not hybrid).
    \item \texttt{kyber\_ct} is the Kyber ciphertext encapsulated to the client's public key.
    \item \texttt{chosen\_suite} describes the selected KEM/signature suite and hybrid flag.
    \item \texttt{signature} is reserved for future Dilithium signatures over the transcript.
\end{itemize}

\section{Shared Secrets and Transcript Hash}

The \texttt{Shared} struct contains:

\begin{lstlisting}
pub struct Shared {
    pub classical: Option<[u8; 32]>,
    pub pq: Vec<u8>,
}
\end{lstlisting}

\begin{itemize}
    \item \texttt{classical} is \texttt{Some(shared\_x25519)} in hybrid mode and \texttt{None} otherwise.
    \item \texttt{pq} is the Kyber shared secret.
\end{itemize}

The handshake transcript consists of a deterministic concatenation of:

\begin{enumerate}
    \item Client random.
    \item Server random.
    \item Client and server X25519 public keys.
    \item Client Kyber public key and server Kyber ciphertext.
    \item Encoded capabilities and chosen suite.
\end{enumerate}

This transcript is hashed with SHA-256 and used as the salt for HKDF. The input keying material is constructed by concatenating the classical shared secret (if present) and the PQ shared secret.

% =========================================
%      CHAPTER 5: IMPLEMENTATION IN RUST
% =========================================

\chapter{Implementation in Rust}
\label{chap:implementation}

\section{Project Structure}

The repository contains the following top-level elements:

\begin{itemize}
    \item \texttt{Cargo.toml}: crate metadata and dependencies.
    \item \texttt{src/main.rs}: CLI and entry point.
    \item \texttt{src/net.rs}: networking logic and tunnel orchestration.
    \item \texttt{src/kem.rs}: handshake and hybrid KEM implementation.
    \item \texttt{src/aead.rs}: HKDF-based session keys and AEAD helpers.
    \item \texttt{src/frame.rs}: frame format for encrypted records.
    \item \texttt{src/sig.rs}: Dilithium signature wrapper.
    \item \texttt{src/crypto.rs}: legacy helpers (kept for reference).
    \item \texttt{docs/protocol.md}: informal documentation of the protocol.
\end{itemize}

\section{Command-Line Interface (main.rs)}

The CLI uses the \texttt{clap} crate with subcommands for client and server roles:

\begin{lstlisting}
#[derive(Parser)]
#[command(name = "QuantumSafe-SMB-Link", version,
          about = "Post-quantum SMB tunnel")]
struct Cli {
    #[command(subcommand)]
    command: Command,
}

#[derive(Subcommand)]
enum Command {
    Server(ServerOpts),
    Client(ClientOpts),
}
\end{lstlisting}

Both \texttt{ServerOpts} and \texttt{ClientOpts} (defined with \texttt{\#[derive(Args)]}) include:

\begin{itemize}
    \item A local bind/dial address (\texttt{listen} or \texttt{dial}).
    \item A remote SMB server address (\texttt{forward}).
    \item Paths to Dilithium keys (for future authenticated mode).
    \item Flags for hybrid mode and PQ requirement.
\end{itemize}

The function \texttt{init\_tracing()} configures \texttt{tracing-subscriber} with an \texttt{EnvFilter}, enabling log-level control via environment variables. Keys are loaded with:

\begin{lstlisting}
fn load_key(path: &PathBuf) -> Result<Arc<[u8]>> {
    let bytes = fs::read(path)
        .with_context(|| format!("failed to read key {:?}", path))?;
    Ok(Arc::from(bytes))
}
\end{lstlisting}

\section{Networking and Tunnel Logic (net.rs)}

\subsection{Server Mode}

In server mode, \texttt{run\_server} binds a \texttt{TcpListener} and accepts incoming tunnel connections:

\begin{itemize}
    \item For each accepted connection:
    \begin{enumerate}
        \item Perform the server-side handshake via \texttt{server\_handshake}.
        \item Connect to the target SMB server.
        \item Start bidirectional encrypted pumping of data using \texttt{pump\_streams}.
    \end{enumerate}
\end{itemize}

\subsection{Client Mode}

In client mode, \texttt{run\_client} listens for local SMB connections, and for each:

\begin{enumerate}
    \item Connects to the remote tunnel peer.
    \item Executes \texttt{client\_handshake}.
    \item Pipes SMB data between the local client and the remote tunnel using the derived keys.
\end{enumerate}

\subsection{Handshake Functions}

The networking layer calls into \texttt{kem.rs} as follows:

\begin{itemize}
    \item \texttt{client\_handshake(stream, config)}:
    \begin{enumerate}
        \item Build \texttt{ClientHello} using \texttt{ClientHello::new}.
        \item Send encoded \texttt{ClientHello}.
        \item Receive and decode \texttt{ServerHello}.
        \item Compute \texttt{Shared} secrets and derive \texttt{SessionKeys}.
    \end{enumerate}
    \item \texttt{server\_handshake(stream, config)}:
    \begin{enumerate}
        \item Read and decode \texttt{ClientHello}.
        \item Generate Kyber and X25519 components.
        \item Build \texttt{ServerHello} and send it.
        \item Derive \texttt{Shared} and \texttt{SessionKeys}.
    \end{enumerate}
\end{itemize}

The handshake functions also compute associated data (AAD) for AEAD, typically including a hash of the transcript to bind the data channel to the negotiated keys.

\subsection{Framed Pumping}

The function \texttt{pump\_streams} splits the tunnel and target TCP streams into read/write halves and concurrently:

\begin{itemize}
    \item Reads plaintext from the SMB side, encrypts it into a frame, and writes it to the tunnel.
    \item Reads encrypted frames from the tunnel, decrypts them, and writes plaintext to the SMB side.
\end{itemize}

Nonce management is handled by \texttt{NonceCounter} from \texttt{aead.rs}, ensuring that each frame uses a unique nonce.

\section{Authenticated Encryption and Key Schedule (aead.rs)}

\subsection{SessionKeys and Roles}

The \texttt{SessionKeys} struct stores transmit (tx) and receive (rx) keys:

\begin{lstlisting}
#[derive(Clone)]
pub struct SessionKeys {
    pub tx: [u8; 32],
    pub rx: [u8; 32],
}

#[derive(Clone, Copy, Debug, Eq, PartialEq)]
pub enum SessionRole {
    Client,
    Server,
}
\end{lstlisting}

The function \texttt{derive\_session\_keys(shared, transcript\_salt, role)}:

\begin{itemize}
    \item Builds input keying material by concatenating classical and PQ secrets.
    \item Initializes HKDF with the transcript salt and IKM.
    \item Expands to two 256-bit keys labeled \texttt{"QuantumSafe tx"} and \texttt{"QuantumSafe rx"}.
    \item Swaps roles so that the client's \texttt{tx} matches the server's \texttt{rx}, and vice versa.
\end{itemize}

\subsection{NonceCounter}

\texttt{NonceCounter} wraps a 64-bit counter and exposes:

\begin{itemize}
    \item \texttt{next()} $\rightarrow$ 96-bit nonce built from the counter and fixed prefix.
\end{itemize}

This yields a unique nonce per frame until the counter wraps (an event considered beyond practical usage limits).

\subsection{AEAD Wrapper Functions}

The functions:

\begin{lstlisting}
pub fn aead_seal(key: &[u8; 32], nonce12: &[u8; 12],
                 plaintext: &[u8], aad: &[u8]) -> Result<Vec<u8>>

pub fn aead_open(key: &[u8; 32], nonce12: &[u8; 12],
                 ciphertext: &[u8], aad: &[u8]) -> Result<Vec<u8>>
\end{lstlisting}

wrap the underlying \texttt{ChaCha20Poly1305} implementation, mapping errors into \texttt{anyhow::Error} with clear messages.

\section{Framing (frame.rs)}

The framing layer defines a record format:

\begin{center}
\begin{tabular}{@{}lll@{}}
    \toprule
    Field & Type      & Description \\
    \midrule
    \texttt{len}   & \texttt{u32} LE & Total length of \texttt{nonce} + \texttt{ct} \\
    \texttt{nonce} & \texttt{u64} LE & Per-frame nonce (counter) \\
    \texttt{ct}    & bytes           & Ciphertext from AEAD \\
    \bottomrule
\end{tabular}
\end{center}

\texttt{write\_frame} and \texttt{read\_frame} are implemented in terms of \texttt{AsyncWriteExt} and \texttt{AsyncReadExt} and bail out on short reads or invalid lengths.

\section{Dilithium Signature Helpers (sig.rs)}

\texttt{sig.rs} provides a thin abstraction over the \texttt{dilithium2} crate:

\begin{lstlisting}
pub fn sign_detached(msg: &[u8], sk_bytes: &[u8]) -> Result<Vec<u8>> { ... }

pub fn verify_detached(msg: &[u8], sig: &[u8], pk_bytes: &[u8]) -> Result<()> { ... }
\end{lstlisting}

A unit test checks round-trip correctness, verifying that a signature on \texttt{"QuantumSafe-SMB-Link"} under a freshly generated keypair will be accepted.

In future iterations, these functions can be used to sign the handshake transcript, providing mutual authentication.

\section{Legacy Crypto Helpers (crypto.rs)}

\texttt{crypto.rs} contains earlier AEAD and HKDF helper functions that are superseded by \texttt{aead.rs}. They serve as reference for the evolution of the design and are kept for completeness.

% =========================================
%           CHAPTER 6: PERFORMANCE
% =========================================

\chapter{Performance Considerations}
\label{chap:performance}

\section{Benchmarking Methodology}

To evaluate the performance of \qsl{}, the following methodology is suggested:

\begin{itemize}
    \item Run micro-benchmarks for:
    \begin{itemize}
        \item \Kyber{} key generation, encapsulation, and decapsulation.
        \item X25519 key generation and shared-secret computation.
        \item AEAD sealing and opening for typical SMB frame sizes (e.g., 4~KiB, 16~KiB).
    \end{itemize}
    \item Measure end-to-end handshake latency between tunnel endpoints.
    \item Measure SMB file transfer throughput over:
    \begin{enumerate}
        \item Raw TCP (baseline).
        \item Classical-only (if implemented).
        \item Hybrid mode (\XDH{} + \Kyber{}).
        \item Pure \Kyber{} mode.
    \end{enumerate}
\end{itemize}

\section{Example Result Tables}

\subsection{Primitive-Level Benchmarks}

\begin{table}[H]
    \centering
    \caption{Example primitive-level benchmark structure (fill with measured values).}
    \begin{tabular}{@{}lrr@{}}
        \toprule
        Operation & Mean time (\si{\micro\second}) & Std.\ dev. (\si{\micro\second}) \\
        \midrule
        Kyber-768 KeyGen    & -- & -- \\
        Kyber-768 Encaps    & -- & -- \\
        Kyber-768 Decaps    & -- & -- \\
        X25519 KeyGen       & -- & -- \\
        X25519 SharedSecret & -- & -- \\
        \ChaCha{} Seal 4~KiB & -- & -- \\
        \ChaCha{} Open 4~KiB & -- & -- \\
        \bottomrule
    \end{tabular}
    \label{tab:primitive-bench}
\end{table}

\subsection{Handshake and SMB Throughput}

\begin{table}[H]
    \centering
    \caption{Example handshake and throughput benchmark structure.}
    \begin{tabular}{@{}lrr@{}}
        \toprule
        Mode & Handshake latency (ms) & Throughput (MB/s) \\
        \midrule
        Raw TCP                & 0   & -- \\
        Classical-only tunnel  & --  & -- \\
        Hybrid (\XDH{} + \Kyber{}) & --  & -- \\
        Pure \Kyber{}          & --  & -- \\
        \bottomrule
    \end{tabular}
    \label{tab:throughput-bench}
\end{table}

You can fill these tables with actual values obtained from experiments.

\section{Qualitative Discussion}

Even without precise numbers, we can predict several trends:

\begin{itemize}
    \item \textbf{Handshake overhead:} Post-quantum key establishment adds computational cost and increases handshake message sizes due to larger keys and ciphertexts. However, this cost is paid only once per connection.
    \item \textbf{Steady-state performance:} After the handshake, performance is dominated by \ChaCha{} AEAD and I/O. The PQ component does not affect steady-state throughput.
    \item \textbf{Hybrid vs PQ-only:} Hybrid mode performs both X25519 and \Kyber{} operations, roughly doubling the cost of key establishment. This is acceptable for connection-oriented SMB usage but should be acknowledged.
\end{itemize}

% =========================================
%           CHAPTER 7: SECURITY
% =========================================

\chapter{Security Analysis}
\label{chap:security}

\section{Security Goals Recap}

The handshake aims to provide:

\begin{itemize}
    \item Confidentiality and integrity of SMB data via AEAD.
    \item Post-quantum security of session keys (due to \Kyber{}).
    \item Forward secrecy via ephemeral keys.
    \item Resistance to downgrade attacks.
\end{itemize}

\section{Hybrid Security}

In hybrid mode, the session key is derived from both classical and post-quantum shared secrets:

\[
    \textsf{IKM} = ss_{\text{classical}} \,\|\, ss_{\text{pq}}.
\]

HKDF with a transcript-based salt extracts a pseudorandom key used to derive traffic keys. Assuming HKDF behaves as a secure key derivation function and at least one of the shared secrets remains unpredictable to the adversary, the derived keys remain secure.

Thus, hybrid mode provides:

\begin{itemize}
    \item Classical security if \Kyber{} is later broken but X25519 remains secure.
    \item Post-quantum security if X25519 is broken (e.g., by quantum computers) but \Kyber{} remains secure.
\end{itemize}

\section{Transcript Binding and Downgrade Resistance}

Because the transcript hash includes:

\begin{itemize}
    \item Both parties' random nonces.
    \item All public keys and ciphertexts.
    \item The encoded capabilities and chosen suite.
\end{itemize}

any modification to these elements by an active attacker will lead to mismatched transcripts and thus mismatched session keys. As a result, AEAD decryption will fail and the connection will not succeed.

The \texttt{pq\_required} flag further limits downgrade risk: if a client requires PQ, the server must choose a suite with a PQ-capable KEM, or abort.

When Dilithium signatures are incorporated over the handshake transcript, an active attacker will be cryptographically prevented from altering the transcript without detection. This is discussed as future work.

\section{Forward Secrecy}

Forward secrecy is achieved through:

\begin{itemize}
    \item Fresh X25519 ephemeral secrets per handshake.
    \item Fresh \Kyber{} keypairs per handshake (as implemented in \texttt{kem.rs}).
\end{itemize}

Compromise of long-term keys (e.g., Dilithium signing keys) does not reveal past session keys, assuming ephemeral secrets are erased after use.

\section{Implementation Considerations}

\subsection{Randomness}

All keys and nonces use \texttt{OsRng}. The security of the protocol relies on the OS-provided CSPRNG. In production, entropy sources and failure modes should be carefully audited.

\subsection{Error Handling and Side Channels}

The use of \texttt{anyhow} yields user-friendly error messages. Care should be taken not to expose internal state or sensitive values in error messages in production. Also, while the underlying PQ libraries aim to be side-channel resistant, a full side-channel analysis is outside the scope of this course project.

\subsection{Denial-of-Service}

Because the server allocates buffers and performs expensive operations after reading the handshake, large or malformed messages could be used for DoS attacks. The implementation mitigates this with \texttt{MAX\_HANDSHAKE\_LEN}, rejecting handshakes exceeding a reasonable size.

% =========================================
%           CHAPTER 8: CONCLUSION
% =========================================

\chapter{Conclusion and Future Work}
\label{chap:conclusion}

This report has presented \qsl{}, a Rust-based prototype of a post-quantum hybrid tunnel for SMB. The project demonstrates how NIST-standardized primitives such as \Kyber{} and \Dilithium{} can be integrated into a practical handshake and key schedule, complementing classical X25519 to provide hybrid security.

From a migration perspective, the design shows that:

\begin{itemize}
    \item Hybrid key establishment can be cleanly added around existing application protocols like SMB without modifying SMB itself.
    \item Capability negotiation and a transcript-based key schedule offer a path to incremental deployment and downgrade resistance.
    \item The implementation overhead is modest, especially in Rust, where strong type-safety and crate ecosystems simplify cryptographic plumbing.
\end{itemize}

Future work includes:

\begin{itemize}
    \item Integrating Dilithium signatures into the handshake for full authentication.
    \item Extending the capability negotiation to support multiple PQ parameter sets and other KEM/signature families.
    \item Running extensive benchmarks across platforms and SMB workloads, and optimizing TCP buffering and framing.
    \item Hardening the codebase with configuration files, logging policies, and tests against malformed inputs.
\end{itemize}

\clearpage

% =========================================
%               mainGRAPHY
% =========================================

\begin{thebibliography}{9}

\bibitem{nist-mlkem}
NIST.
\emph{FIPS 203: Module-Lattice-Based Key-Encapsulation Mechanism Standard}.
2024.

\bibitem{nist-mldsa}
NIST.
\emph{FIPS 204: Module-Lattice-Based Digital Signature Standard}.
2024.

\bibitem{bernstein-chacha20}
D.~J. Bernstein.
\emph{The ChaCha20-Poly1305 Authenticated Encryption}.
2015.

\bibitem{rfc8439}
Y.~Nir and A.~Langley.
\emph{ChaCha20 and Poly1305 for IETF Protocols}.
RFC 8439, 2018.

\bibitem{x25519}
D.~J. Bernstein.
\emph{Curve25519: high-speed elliptic-curve Diffie-Hellman}.
2006.

\end{thebibliography}



% =========================================
%                 APPENDICES
% =========================================

\appendix

% -----------------------------------------
%           APPENDIX A: WIRE FORMAT
% -----------------------------------------

\chapter{Wire Formats}
\label{app:wire}

This appendix specifies the on-the-wire layout of all handshake messages.

\section{ClientHello}

\begin{center}
\begin{tabular}{@{}lll@{}}
    \toprule
    Field & Type         & Description \\
    \midrule
    Magic         & 4 bytes     & \texttt{"QSLH"} \\
    Version       & u16 (BE)    & Protocol version (1) \\
    Length        & u16 (BE)    & Payload length \\
    Random        & 32 bytes    & Client random nonce \\
    X25519 pk     & 32 bytes    & Client public key (all zeros if not hybrid) \\
    Kyber pk len  & u16 (BE)    & Length of Kyber public key \\
    Kyber pk      & variable    & Kyber public key bytes \\
    Caps len      & u16 (BE)    & Length of capabilities TLV block \\
    Caps (TLVs)   & variable    & TLVs with KEM list, SIG list, hybrid, pq\_required \\
    \bottomrule
\end{tabular}
\end{center}

\section{ServerHello}

\begin{center}
\begin{tabular}{@{}lll@{}}
    \toprule
    Field & Type         & Description \\
    \midrule
    Magic           & 4 bytes   & \texttt{"QSLS"} \\
    Version         & u16 (BE)  & Protocol version (1) \\
    Length          & u16 (BE)  & Payload length \\
    Random          & 32 bytes  & Server random nonce \\
    X25519 pk       & 32 bytes  & Server X25519 public key (zeros if not hybrid) \\
    Kyber ct len    & u16 (BE)  & Length of Kyber ciphertext \\
    Kyber ct        & variable  & Kyber ciphertext \\
    Suite len       & u16 (BE)  & Length of chosen-suite encoding \\
    Chosen suite    & variable  & Encoded KEM+SIG+hybrid selection \\
    Sig len         & u16 (BE)  & Length of signature (0 if unused) \\
    Signature       & variable  & Dilithium signature over transcript (future) \\
    \bottomrule
\end{tabular}
\end{center}

\section{Encrypted Data Frame}

Data frames follow the format described in \Cref{chap:implementation}: \texttt{len} (u32 LE), \texttt{nonce} (u64 LE), and ciphertext bytes.

% -----------------------------------------
%          APPENDIX B: CLI USAGE
% -----------------------------------------

\chapter{CLI Usage}
\label{app:cli}

\section{Server Mode Example}

\begin{verbatim}
QuantumSafe-SMB-Link server \
    --listen 0.0.0.0:4444 \
    --forward 10.0.0.5:445 \
    --hybrid \
    --pq-required
\end{verbatim}

\section{Client Mode Example}

\begin{verbatim}
QuantumSafe-SMB-Link client \
    --listen 127.0.0.1:1445 \
    --dial 192.0.2.10:4444 \
    --hybrid \
    --pq-required
\end{verbatim}

In this configuration, SMB clients can connect to \texttt{127.0.0.1:1445}, and their traffic will be tunneled securely to the SMB server at \texttt{10.0.0.5:445} via the post-quantum-aware tunnel at \texttt{192.0.2.10:4444}.

\end{document}
